\documentclass{article}
\usepackage{graphicx}
\usepackage{fancyhdr}
\usepackage{ifthen}
\usepackage{float}
\usepackage{amsmath}
\usepackage[bottom=1in]{geometry} 
\usepackage{hyperref}
\usepackage{amssymb}

\fancypagestyle{plain}{
    \fancyhf{}
    \fancyheadoffset[L]{+2cm}
    \fancyhead[L]{{Métodos Númericos y Optimización TP2 2024 {1-\pageref{LastPage}}}}
    \fancyheadoffset[R]{+2cm}
    \fancyhead[R]{Terminado 17/5/2024, Publicado 18/5/2024}
    \renewcommand{\headrulewidth}{0pt}
    \renewcommand{\footrulewidth}{0pt}
    \fancyfoot[C]{\thepage}
}

\fancypagestyle{myheader}{
    \fancyhf{}
    \fancyhead[LE,RO]{\ifthenelse{\isodd{\value{page}}}{Analisis de variacion poblacional en competencia intraespecifica, interespecifica y depredación de Lotka Volterra}{Manuel Meiriño y Felicitas Marolda}}
    \fancyfoot[C]{\thepage}
}

\pagestyle{myheader}

\makeatletter
\renewcommand{\maketitle}{
    \begin{center}
        {\huge \@title}\\
        \vspace{10pt} 
        {\Large \@date}
    \end{center}
}
\makeatother

\title {Analisis de variacion poblacional en competencia intraespecifica, interespecifica y depredación de Lotka Volterra}
\date{Mayo 2024}

\begin{document}

\pagenumbering{arabic}

\maketitle
\thispagestyle{plain}

\begin{center}

\large
\textbf{Marolda, Felicitas}\\
Universidad de San Andres\\
fmarolda@udesa.edu.ar\\
N° de Legajo: 35494\\
\vspace{1\baselineskip}

\large
\textbf{Meiriño, Manuel}\\
Universidad de San Andres\\
mmeirino@udesa.edu.ar\\
N° de Legajo: 35723
\end{center}

\vspace{0.5cm}

\section*{Abstract}
\normalsize

% \noindent Compararemos la eficiencia de dos de estos metodos, analizaremos como diferencias en ciertos parámetros cambian el crecimiento poblacional, tanto en modelos con 1 especie, como con 2 especies compitiendo por recursos, y 2 especies en relacion de depredación. Y por ultimo, evaluaremos las isoclinas y puntos de equilibrio que pueden deducirse a partir de dichas funciones, sus variaciones y su significado.

\vspace{1\baselineskip}





\section*{Introducción}

\noindent 

 
 \noindent 
% que son las ODEs
% humo
% objetivo de nuestro trabajo es
Las Ecuaciones Diferenciales Ordinarias (EDOs, o en ingles, ODEs) son ecuaciones en las cuales se ve incluida una derivada de la funcion en la igualdad. Dichas ecuaciones pueden despejarse para representar variacion (con la derivada). \vspace{\baselineskip}

\noindent Existen casos en los que la ODE puede ser analíticamente resuelta para encontrar la función primitiva, pero también casos en los cuales esto es extremadamente complejo, o simplemente imposible, y en dichos casos se pueden aplicar ciertos metodos numericos para, utilizando la ODE y un valor inicial, aproximar valores que tomaria la funcion 'primitiva' (o mejor dicho, la funcion de la cual las ODEs representan la variacion) a traves de una cierta variable. \vspace{\baselineskip}


\noindent Las isoclinas de las ODEs son las curvas en las cuales el gradiente es el mismo. Las isoclinas cero de las ODEs, mas especificamente, son las curvas donde la ODE es = 0. (University of Nebreaska-Lincoln, 2009). Al analizar un sistema de ODEs, se le llama punto de equilibrio al punto en el cual las ODEs del sistema son todas iguales a 0, estos puntos existen en la interseccion de sus isoclinas cero, y posiblemente en mas posiciones dependiendo de la ecuacion especifica. Los puntos de equilibrio son considerados estables cuando los sistemas inicializados van a tener una trayectoria que tiende hacia dicho punto, y si sucede lo contrario (si las trayectorias se alejan del punto), se dice que el punto es inestable (University of Wyoming, 2013). \vspace{\baselineskip}

\noindent El objetivo de esta investigacion es usar dos metodos numericos y (en algunos casos) soluciones analiticas para evaluar las variaciones en crecimiento poblacional en 3 distintos modelos: 1 especie; 2 especies compitiendo por recursos; y 2 especies en relacion de depredacion. A la vez evaluando la eficiencia de estos metodos, y discutir sobre la informacion que proveen las isoclinas cero y puntos de equilibrios de las ODEs.

\vspace{0.5\baselineskip}

\section*{Materiales y Métodos}

\noindent Los materiales utilizados en esta investigacion consisten principalmente en nuestras computadoras para escribir y ejecutar los codigos, usando el programa de Visual Studio Code. Se utilizaron las librerias numpy (para expresiones numericas) y matplotlib para graficar los resultados en cada función. Se utilizó Microsoft Word para las referencias.

\subsection*{Método de Euler}

\noindent El método de Euler es un método numérico utilizado para resolver ecuaciones diferenciales. Dado una ODE y' y una condición inicial $y_{0}$, se busca aproximar y(t) en un intervalo específico [a,b]. En vez de resolverla de forma analítica, se obtienen distintos valores de $t \in [a,b]$ equiespaciados. Los valores de t van a estar separados por h que se calcula haciendo $(a-b)/n$ de manera tal que $h = t_{i+1} - t_{i}$. Utilizando estos valores el método de Euler utiliza la siguiente fórmula:

\setlength{\baselineskip}{0.75\baselineskip}
\[ w_{0} = y_{0} \]
\[ w_{i+1} = w_{i} + h \cdot f(t_{i}, w_{i}) \]
\setlength{\abovedisplayskip}{-6pt}
\vspace{0.5\baselineskip}

\noindent Con el método de Euler, estamos suponiendo que al calcular $w_{i}$ para distintos t se debe cumplir que $f(t_{i},w_{i}) \approx y'(t_{i}) = f(t_{i},y(t_{i}))$. (Burden and Faires, 2011)

\subsection*{Método de Runge-Kutta 4}

\noindent Similarmente al método de Euler, Runge-Kutta 4 también busca aproximar la resolución de ecuaciones diferenciales dada una ODE y una condición inicial. La fórmula utilizada es la siguiente:

\setlength{\baselineskip}{0.75\baselineskip}
\[ w_{0} = y_{0} \]
\[k_{1} = hf(t_{i}, w_{i})\]
\[k_{2} = hf(t_{i} + \frac{h}{2}, w_{i} + \frac{k_{1}}{2})
\]
\[
k_{3} = hf(t_{i} + \frac{h}{2}, w_{i} + \frac{k_{2}}{2})
\]
\[
k_{4} = hf(t_{i} + h, w_{i} + k_{3})\]
\[w_{i+1} = w_{i} + \frac{1}{6}(k_{1} + 2k_{2} + 2k_{3} + k_{4})
\]

\setlength{\abovedisplayskip}{-6pt}

\noindent Para aplicar Runge-Kutta creamos funciones auxiliares que reciben los parámetros requeridos dependiendo de la actividad, una ecuación diferencial y un intervalo de tiempo. De esta manera pudimos analizar el impacto de los diferentes parámetros en la aproximación de cada sistema de ODEs.


\section*{Desarrollo Experimental}


\subsection*{Actividad 1: Competencia Intraespecifica}

\noindent Comenzamos esta investigación con una especie, la cual en este caso serán los pandas rojos (quienes serán referidos como 'pandas' por simpleza). Y el primer modelo estudiado en estos pandas fue el de la competencia intraespecifica, analizando la evolución poblacional de los pandas cuando estos viven solos en su propio ambiente. \vspace{1\baselineskip}

\noindent Hay 2 casos a evaluar, uno en el cual los pandas viven con recursos ilimitados, y tienen su crecimiento de población exponencial sin limites, por el otro lado esta el caso en el cual los pandas tienen que vivir y reproducirse con recursos limitados, lo cual actúa como limitante de su crecimiento, a este segundo se le llama modelo logístico. \vspace{1\baselineskip}

\noindent Utilizando los parámetros N (población) y r (tasa de crecimiento o growth rate), la variación de población en el modelo exponencial se representa con la siguiente ecuación diferencial: \vspace{0.5\baselineskip}

\[\frac{dN}{dt} = rN \]

\vspace{0.75\baselineskip}

\noindent Por el otro lado, agregando el parámetro K (capacidad de carga) al sistema, el modelo logístico se representa con la siguiente ODE: \vspace{0.5\baselineskip}

\[\frac{dN}{dt} = rN \frac{K - N}{K}\]
\vspace{0.75\baselineskip}

\noindent Ya que estas soluciones tienen una resolución analítica realizables sin complejidad muy alta, fueron resueltas analíticamente (ver Apéndice 1.1), y se probaron variaciones en los parámetros para ver como resultaban en cambios de la población a través del tiempo para cada modelo. Y luego se calculo como variaba el crecimiento (el valor de cada ODE) con respecto a su respectiva primitiva (Variación de Población en base a Población), evaluando en diferentes parámetros. \vspace{1\baselineskip}

\noindent Por ultimo, se utilizaron los ya-mencionados métodos numéricos de Euler y Runge Kutta, y se aproximaron los valores de poblacion en ambos modelos a traves del tiempo, probando como cambiaban al variar ciertos parametros, y tambien comparando con los resultados analiticos para determinar que metodo era mas efectivo. Las aproximaciones se realizaron usando un tiempo de 50, con 100 steps de 0.5 cada uno (se decidio un numero de steps no tan alto, para reflejar mejor que tan eficiente era cada aproximacion).



\subsection*{Actividad 2: Competencia Interespecifica}
\noindent Para esta parte de la investigacion, ahora los pandas tendran que convivir con otra especie, las tortugas, que tambien son herbivoras y competiran por recursos con los pandas. En este modelo, la variacion poblacional de cada especie fue representada con el modelo de competencia interespecifica de Lotka Volterra.\vspace{1\baselineskip}

\noindent Usando las variables $N_p$ (Poblacion de Pandas), $N_t$ (Poblacion de Tortugas); $r_p$ (Tasa de Crecimiento o 'Growth Rate' de Pandas); $r_t$ (Growth Rate de Tortugas); $K_p$ (Capacidad de Carga de Pandas); $K_t$ (Capacidad de Carga de Tortguas); $a_p$ (Coeficiente de Competencia de Pandas); $a_t$ (Coeficiente de Competencia de Tortugas), dicho modelo se represento con las siguientes ecuaciones diferenciales: \vspace{0.5\baselineskip}

\setlength{\baselineskip}{1\baselineskip}
\[\frac{dN_p}{dt} = r_p N_p \left( \frac{K_p - N_p - \alpha_{t} N_t}{K_t} \right)\]
\[\frac{dN_t}{dt} = r_t N_t \left( \frac{K_t - N_t - \alpha_{p} N_p}{K_t} \right)\]
\vspace{0.75\baselineskip}

\noindent Teniendo esta informacion, para evitar tener que resolver las ecuaciones analiticamente, se utilizo el metodo de Runge Kutta para aproximar las dinamicas de poblacion entre las 2 especies en diversos casos. El caso "normal" es uno en el que los pandas tenian mayor poblacion inicial, mayor tasa de reproduccion y capacidad de carga que las tortugas, sin embargo estas tenian un mayor coeficiente de competencia, en los demas casos se incrementaron los distintos parametros de los pandas, sin tocar los de las tortugas, para contrastar como cambiaban las dinamicas. Todas estas aproximaciones se realizaron en un tiempo de 270 con 2700 steps de 0.1 cada uno.
\vspace{1\baselineskip}

\noindent Luego, para obtener más información de las soluciones estudiamos las isoclinas cero de cada especie (calculadas en Apendice 1.2) y como estas eran afectadas por los cambios en los distintos parámetros, utilizando nuevos casos que mejor representen de forma clara las variaciones que estos calculos generaban, reflejandolo tambien en nuevas aproximaciones de poblacion para cada uno de los graficos de isoclineas. \vspace{1\baselineskip}

\noindent Y utilizando las isoclinas, calculamos los puntos de equilibrio en los distintos casos, graficandolos junto a varias trayectorias y un campo diferencial, lo cual ayudaria a determinar su estabilidad.



\subsection*{Actividad 3: Depredacion}
\noindent En esta ultima parte del experimento, los pandas ahora no conviven y compiten con otra especie herbivora como ellos, sino que tienen que coexistir y sobrevivir en un ambiente con uno de sus depredadores naturales (el Leopardo de las Nieves, quienes seran llamados 'Leopardos' por simpleza). En esta situacion, la variacion de poblacion se representa con el modelo de depredacion de Lotka Volterra:
\vspace{0.5\baselineskip}

\setlength{\baselineskip}{1\baselineskip}
\[\frac{dP}{dt} = rP - \alpha PL \]\\
\[\frac{dL}{dt} = \beta PL - qL\]
\vspace{0.75\baselineskip}

\noindent Al igual que en la actividad 2, se llevó a cabo un análisis del impacto de los parámetros en la solución utilizando métodos numéricos. Similarmente, se analizaron las isoclinas y puntos de equilibrio. \vspace{1\baselineskip}

\noindent A este modelo de predador-presa se le sumó la competencia intraespecífica de las presas, definiendo las ecuaciones de Lotka-Volterra Extendidas (LVE), las cuales se analizaron de igual manera que el previo sistema de ecuaciones diferenciales. \vspace{0.5\baselineskip}

\setlength{\baselineskip}{1\baselineskip}
\[\frac{dP}{dt} = rP - \alpha PL - \frac{rP^2}{K} = rP \left(1 - \frac{P}{K}\right) - \alpha PL \]\\
\[\frac{dL}{dt} = \beta PL - qL\]
\vspace{0.75\baselineskip}

\noindent Utilizando estas ecuaciones, nuevamente se modelaron diversos escenarios en base a variaciones en los parámetros, analizando las consecuencias en los cambios poblacionales de ambas especies en cada caso. %TODO: Agregar algun sentido y proposito a las aproxs, you know
Todas las aproximaciones fueron realizadas usando un tiempo de 150 y 1500 steps de 0.1 cada uno.
\vspace{1\baselineskip}

\noindent Luego se considero graficar las ODEs de ambas especies en ambos sistemas, sin embargo antes de realizarla se descartó esa opción como algo util o informativo. \vspace{1\baselineskip}

\noindent Se dedujo que dicha tarea no se podría realizar en un grafico de dimension \( R^2 \) como los otros, debido a que estaria teniendo en cuenta por lo menos 3 ejes de informacion (Poblacion de Presas, Poblacion de Depredadores, Variacion de Poblacion) ya que la variacion de ambos depende tanto de su propia poblacion como de la poblacion de la otra especie, por lo cual graficar la variacion de cada especie en base a su propia poblacion implicaria necesariamente fijar un valor de la poblacion de la otra especie, los graficos tendrian que presentarse en \( R^3 \) dimensiones, perdiendo su intuicion visual para transmitir informacion util. \vspace{1\baselineskip}

\noindent Por ultimo para estos modelos, se graficaron las isoclinas en cada uno de los sistemas, cada una junto con un grafico representando una trayectoria de ejemplo para visualizar mejor las tendencias de ambas especies en cada uno de los sistemas. Las trayectorias de ejemplo fueron aproximadas usando un tiempo de 150, y 1500 steps de 0.1 cada uno.





\section*{Resultado y Análisis}



\subsection*{Actividad 1}
\noindent cosaaaaaaaaaaaas

% ACTIVIDAD 1
% ACORDATE DEL ERROR, EL ERROR MENOR DE RK DETERMINA USAR RK > USAR EULER

\noindent Los resultados de error relativo promedio usando 

Average Relative Error Euler (Exponential): 0.2014
Average Relative Error Runge-Kutta (Exponential): 0.0437

Average Relative Error Euler (Logistique): 0.0555
Average Relative Error Runge-Kutta (Logistique): 0.0133








\subsection*{Actividad 2}
\noindent Luego de resolver las ecuaciones diferenciales en la actividad 1 concluimos que Runge-Kutta 4 era más preciso, por esto, utilizamos este método numérico para resolver el modelo de Lotka-Volterra en esta actividad. 
\vspace{1\baselineskip}
\vspace{1\baselineskip}



%Runge kutta y el cambio de parámetros en la actividad 2 con sus gráficos
\noindent HABLAR DEL RUNGE-KUTTA EN EL 2
\vspace{1\baselineskip}



\noindent Después de resolver con Runge-Kutta calculamos las isoclinas igualando las ecuaciones diferenciales a 0 y despejando N1 y N2.
\vspace{0.5\baselineskip}

\begin{equation}
\text{Isoclina $N_p$ = } \frac{K_p}{\alpha_{p}} - \frac{1}{\alpha_{p}} \cdot N_p
\end{equation}

\begin{equation}
\text{Isoclina $N_t$ = } K_t - \alpha_{t} \cdot N_p
\end{equation}
\vspace{0.75\baselineskip}

\noindent Como se puede ver, las pendientes están definidas por $\alpha_p$ y $\alpha_t$, por lo cual si los Coeficientes son inversos ($\alpha_p$ = $\frac{1}{\alpha_t}$), entonces la pendiente de ambas isoclinas será igual, por ende, nunca se cruzaran y la isoclina que tenga el valor K más grande va a encontrarse siempre por encima de la otra. Esto se puede ver en los siguientes gráficos.

\begin{figure}[ht]
    \centering
    \caption{panchos}
    \includegraphics[width=0.6\linewidth]{}
    \label{fig:Image 1.1}
\end{figure}



%graficos de isoclinas sin interseccion

%hablar de los puntos de equilibrio y que pasa con estos cuando son "paralelas" las isoclinas

% Results 



% ACTIVIDAD 2. 
% Lo importante es que en el default, Turtles tiene un poco mas alto indice de competencia, y que los graficos 2 y 3 muestran que incluso aumentando bastante el growth rate de los pandas o la initial pop de los pandas, las turtles still ganan con su mejor competencia. Sin embargo, cuando se aumenta la capacidad de carga de los pandas, le ganan a las turtles (y por supuesto, cuando se incrementa la competencia de los pandas, arrasan as expected).

% En RK de Act2, aclarar que estos graficos todos muestran 
% tal valor de pandas > tal valor de turtles
% Pero estos graficos podrian ser iguales invertidos, solo estamos eligiendo mostrarlo asi para no llenar demasiado espacio, si se hace     a2 > a1, va a ser lo mismo que a1 > a2 pero con las especies invertidas

% En los graficos de isoclinas
% trayectoria ejemplo means con los datos del que esta mostrado al lado, con la inicial de pandas en 100 y la de tortugas en 60 as always.



% Comments Sobre Isoclinas Act 2

% En el primer grafico claramente gana la tortuga, pto de equilibrio es pandas = 0, tortugas = Kt

% En el segundo grafico claramente gana el panda, pto de equilibrio es pandas = kp, tortugas = 0

% En el tercer grafico fijate que si empezas en ciertos puntos gana la tortuga (p=0, t=Kt) pero en otros gana el panda (p=Kp, t=0). El de la isct es pto de equi porque ambas ODEs son 0, pero no es estable porque ninguna trayectoria tiende al punto, se alejan

% En el cuarto grafico todas las trayectorias tienden al equilibrio.






\subsection*{Actividad 3}
\noindent 
%graficos de runge kutta y hablar de como afecta el cambio de los diferentes parámetros y que cambios no afectan


\noindent Para tener una mayor comprensión de lo que sucedía y de las soluciones a estos sistemas de ecuaciones diferenciales calculamos las isoclinas del modelo de Predador-Presa de Lotka-Volterra como se puede ver a continuación:
\vspace{1\baselineskip}

\begin{align*}
\text{isoclina}_{N}: \frac{q}{\beta} \\
\text{isoclina}_{P}: \frac{r}{\alpha}
\end{align*}

\noindent De estas ecuaciones pudimos ver que ninguna se ve afectada por la otra y que sus valores solo dependen de los parámetros correspondientes; la isoclina de N esta definido por $\beta$ y $q$, mientras que la isoclina de P se ve definida por $r$ y $\alpha$. Esto indica que las isoclinas son constantes, y esto fue lo que se vio en el gráfico. Por esto mismo, a diferencia de la actividad 2, no habrán diferentes casos para las isoclinas y estabilidad de este modelo ya que los valores de los parámetros no alteran la forma de las isoclinas, sino simplemente su posición en el plano de fases. En consecuencia, el análisis de estabilidad y el comportamiento cualitativo de las trayectorias será esencialmente el mismo independientemente de los valores numéricos específicos de los parámetros.
\vspace{1\baselineskip}

\vspace{10\baselineskip}
\noindent Aca va el grafico de las isoclinas Depredacion
\vspace{10\baselineskip}


\vspace{1\baselineskip}

\noindent En el grafico de la izquierda se puede observar que las isoclinas del modelo de Predador-Presa de Lotka-Volterra son rectas constantes. Aunado a que las trayectorias en el plano Presas vs Depredadores tienen forma de órbitas circulares alrededor del punto de equilibrio, esto implica que el sistema oscila indefinidamente sin converger al equilibrio. Esto se debe a la ausencia de términos no lineales que introduzcan efectos estabilizadores en las trayectorias. Es decir, no hay términos que hagan que las trayectorias converjan hacia el punto de equilibrio. Dicha osiclacion indefinida se puede apreciar en el grafico de la derecha, en el cual las trayectorias nunca parecen converger hacia ningun equilibrio definido.
\vspace{1\baselineskip}


\noindent Para contrastar, hacemos el mismo análisis del modelo de Lotka-Volterra extendido. En este caso, se ha introducido un nuevo término cuadrático en la ecuación de las presas, que representa la competencia intraespecífica. Esto modifica la forma de las ecuaciones diferenciales, las cuales ahora son dependientes entre sí y no lineales. Por lo tanto, es de esperar que las isoclinas y las trayectorias no sean constantes y tengan un comportamiento diferente al modelo original. Para formalizar este análisis, calculamos las isoclinas:

\begin{align*}
\text{isoclina}_{N}: &  \frac{q}{\beta} \\
\text{isoclina}_{P}: & \frac{r}{\alpha}\left(1 - \frac{N}{K}\right)
\end{align*}

\noindent Similarmente al modelo de Predador-Presa de Lotka-Volterra acá tampoco habrán diferentes casos. Esto sucede porque para que el modelo tenga sentido la intersección de las rectas debería observarse en el primer cuadrante y esto solo sucede cuando $K > \frac{q}{\text{beta}}$. Debido a esto, y a las restricciones de los otros parámetros, solo quedará un caso en el que el punto de equilibrio se encuentra en el primer cuadrante.  %quizas habra que chequear esto un poco o elaborar más

\vspace{1\baselineskip}
\noindent A diferencia del modelo original, las isoclinas ahora no son rectas constantes, sino que dependen de las variables de población $N$ y $P$. Esto implica que las trayectorias en el plano de fases tendrán un comportamiento diferente, como se analizará a continuación.

\vspace{1\baselineskip}
%quizas decir algo ma s del punto de equilibrio

\noindent En el modelo de Lotka-Volterra extendido (LVE), que incorpora la competencia intraespecífica de las presas, las trayectorias en el plano Presas vs Depredadores ya no son órbitas circulares perfectas. En su lugar, adquieren una forma de espiral que eventualmente converge al punto de equilibrio del sistema. Esta diferencia en la forma de las trayectorias se debe a la modificación introducida en la ecuación de las presas por el término de competencia intraespecífica.
\vspace{1\baselineskip}

\vspace{10\baselineskip}
\noindent Aca va el grafico de las isoclinas LVE
\vspace{10\baselineskip}
\vspace{1\baselineskip}

\noindent En esta figura se puede observar el comportamiento en espiral mencionado anteriormente. Se pueden apreciar que las trayectorias, dibujadas con líneas discontinuas negras, tienen forma de espiral y convergen hacia el punto de equilibrio, que es la intersección de las isoclinas.
\vspace{1\baselineskip}

\noindent Este comportamiento en espiral implica que, independientemente de las condiciones iniciales, el sistema eventualmente alcanzará un estado estacionario en el punto de equilibrio. Esto se puede confirmar al analizar la evolución temporal de las poblaciones, como se muestra en la Figura XXX.
\vspace{1\baselineskip}

\noindent Como se ve en el cuadro de la derecha, las poblaciones de presas y depredadores comienzan oscilando, pero eventualmente convergen a valores constantes que corresponden al punto de equilibrio del sistema. Por lo tanto, la incorporación del término de competencia intraespecífica en el modelo LVE introduce un comportamiento estable y convergente en el sistema, a diferencia del modelo original de Lotka-Volterra, donde las trayectorias eran órbitas circulares que nunca convergían al equilibrio.
\vspace{1\baselineskip}

\noindent Esta diferencia en la forma de las trayectorias y el comportamiento a largo plazo del sistema es una consecuencia directa de la modificación en la ecuación de las presas, lo que demuestra la importancia de considerar la competencia intraespecífica al modelar sistemas ecológicos.
\vspace{1\baselineskip}


%habria que deicr algo mas de poruqe no hay casos











\section*{Conlusión}
\noindent cosaaaaaaaaaaaas



\subsection*{Potenciales Mejoras}

\noindent cosaaaaaaaaaaaas



\label{LastPage}


\section*{Bibliografía}

\noindent cosaaaaaaaaaaaas


\section*{Apéndice}

% 1.1 es Resolucion Analitica de las ODEs de Actividad 1
% 1.2 es Resolucion Analitica de las Isoclinas de Actividad 2

% aca poner los despejes de las ecuaciones



% Grafico de las ODEs en el 3

Las variaciones no se graficaron en base a las poblaciones como se había hecho en la Competencia Intraespecífica, debido a que en este caso eso implicaría graficar en mayores dimensiones para tener datos completos (ya que se usarian 3 ejes: Población de Presas, Población de Depredadores, Variaciones), y una cantidad finita de graficos en dimensión normal en practica no lo cubriría debido a que tan solo seria fijar 1 igualdad en algún/os planos y repetirla (por ejemplo, si se quisiera evaluar en tiempo 100 con step 0.1, tendrían que hacerse 1000 gráficos distintos que fijen alguna condición de igualdad en alguno de los ejes de Población o entre ellos). 



\end{document}